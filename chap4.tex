% Chapter 4
\chapter{تعویض مدل‌های شبکه عصبی بین کاربران}

\section{مقدمه}
در فصل پیشین، روش‌های متعددی برای حل مشکل داده‌های
\lr{non-IID}
مورد بررسی قرار گرفتند. در این فصل، رویکرد جامعی برای مقابله با این چالش، یعنی تبادل مدل‌های شبکه عصبی میان کاربران نهایی، بررسی می‌شود که محور اصلی این پایان‌نامه را تشکیل می‌دهد. به منظور درک بهتر، ابتدا یک مثال از داده‌های
\lr{non-IID}
مطرح خواهد شد.

فرض کنید هدف، آموزش مدلی برای تشخیص اشیا مانند علائم ترافیکی و علائم فروشگاهی است. اگر وسایل نقلیه به ترتیب در بزرگراه و مرکز شهر حرکت کنند، داده‌های ویدیویی آن‌ها توزیع‌های متفاوتی از این علائم خواهند داشت. به این معنا که داده‌های آموزشی جمع‌آوری شده از بزرگراه ممکن است کمتر شامل علائم فروشگاهی باشند، در حالی که داده‌های جمع‌آوری شده از مرکز شهر حاوی تعداد بیشتری از هر دو نوع علائم خواهند بود. این تفاوت در توزیع داده‌ها در دستگاه‌های نهایی می‌تواند باعث ایجاد مشکل انحراف وزن‌ها شود.

برای حل این مسئله، عملیات تعویض مدل‌های شبکه عصبی بین کاربران نهایی پیشنهاد می‌شود. این رویکرد، مدل‌ها را بین دستگاه‌های نهایی جابجا می‌کند تا تنوع داده‌ها در دستگاه‌های مختلف کاهش یابد. این عملیات بدون نیاز به هزینه‌های محاسباتی و ارتباطی اضافی، به بهبود عملکرد مدل در مواجهه با داده‌های
\lr{non-IID}
کمک می‌کند.

در این فصل ابتدا ...

\section{
	روش تعویض فدرال%
\LTRfootnote{Federated Swapping}
}

در این روش، یک عملیات جدید به نام تعویض فدرال یا
\lr{FedSwap}
پیشنهاد شده است که جایگزین برخی از دوره‌های
\lr{FedAvg}
در سرور می‌شود. این عملیات با هدف بهبود فرآیند یادگیری فدرال و کاهش تاثیرات منفی داده‌های
\lr{non-IID}
طراحی شده است. اصل اساسی
\lr{FedSwap}
این است که به جای اجرای
\lr{FedAvg}
در هر تکرار، به دستگاه‌های نهایی اجازه می‌دهد تا مدل‌های محلی خود را در سرور با یکدیگر تبادل کنند
\cite{chiu2020semisupervised}.


در روش اصلی یادگیری فدرال، در هر تکرار یک مدل جهانی جدید از ترکیب مدل‌های هر دستگاه نهایی به دست می‌آید. همان‌طور که در شکل
\ref{federated_swapping}
نشان داده شده است، به جای اجرای
\lr{FedAvg}
در هر تکرار، سرور می‌تواند به عنوان یک گزینه دیگر، مدل‌ها را بین دستگاه‌های نهایی تعویض کند. در این روش، به جای اینکه در هر تکرار فرآیند
\lr{FedAvg}
انجام شود، دستگاه‌های نهایی اجازه دارند مدل‌های محلی خود را در سرور با یکدیگر مبادله کنند.


\begin{figure}[b!]
	\centering
	\includegraphics[scale=0.2]{images/chap4/federated_swapping.png}%
	\caption{%
		روش تعویض فدرال
		\cite{chiu2020semisupervised}%
		.
	}
	\label{federated_swapping}
	\centering
\end{figure}


برای حفظ عدالت در این فرآیند، از یک استراتژی چرخشی استفاده می‌شود. در این استراتژی، به طور منظم و به ترتیب، دو دستگاه نهایی به یکدیگر اجازه می‌دهند که مدل‌های خود را تبادل کنند. این کار باعث می‌شود که همه دستگاه‌های نهایی به طور مساوی در فرآیند تبادل مدل‌ها شرکت کنند و هیچ دستگاهی از مزایای این تبادل محروم نماند.

علاوه بر این، انتظار می‌رود که این عملیات تعویض مدل بین دستگاه‌های نهایی، به هر مدل دید گسترده‌تری از کل مجموعه داده‌ها بدهد. به عبارت دیگر، هر مدل محلی با تبادل مدل با دیگر دستگاه‌ها، می‌تواند اطلاعات بیشتری از داده‌های مختلف دریافت کند. این امر به کاهش انحراف وزن‌ها کمک می‌کند، زیرا مدل‌ها با داده‌های متنوع‌تری آموزش می‌بینند و به تدریج به یک مدل جامع‌تر و دقیق‌تر نزدیک می‌شوند.

به طور خلاصه، عملیات
\lr{FedSwap}
با تبادل مدل‌های محلی بین دستگاه‌های نهایی، نه تنها به بهبود دقت و عملکرد مدل‌ها کمک می‌کند، بلکه مشکلات ناشی از داده‌های
\lr{non-IID}
را نیز کاهش می‌دهد. این روش به عنوان یک رویکرد موثر در یادگیری فدرال می‌تواند باعث بهبود قابل توجهی در نتایج نهایی شود
\cite{chiu2020semisupervised}.


جزئیات عملیات
\lr{FedSwap}
در الگوریتم
\ref{algo_FedSwap}
ارائه شده است.
همچنین در جدول
\ref{tabel_FedSwapNotations}
نمادهای مختص این الگوریتم به نمایش در آمده است.
در این الگوریتم،
$w^k_t$
به عنوان وزن مدل در دستگاه نهایی
$k$
پس از گام
$t$
تنظیم می‌شود. در ابتدا، دستگاه‌های نهایی چندین به‌روزرسانی محلی انجام می‌دهند تا مدل‌های خود را بهبود بخشند. پس از هر
$h_1$
به‌روزرسانی محلی، سرور وارد عمل شده و عملیات
\lr{FedSwap}
را اجرا می‌کند. در این مرحله، مدل‌های محلی بین دستگاه‌های نهایی تبادل می‌شوند تا هر دستگاه بتواند از مدل‌های متنوع‌تری برای آموزش استفاده کند.

این تبادل مدل‌ها به کاهش تنوع داده‌ها بین دستگاه‌های مختلف کمک می‌کند و باعث می‌شود که مدل‌ها با داده‌های مختلفی آموزش ببینند. پس از انجام
$h_2$
عملیات
\lr{FedSwap}%
، سرور وارد عمل شده و عملیات
\lr{FedAvg}
را اجرا می‌کند. در این مرحله، سرور مدل‌های محلی را تجمیع می‌کند تا یک مدل مشترک ایجاد شود که از داده‌های تمام دستگاه‌ها بهره‌مند است.


\begin{LTR}
	\SetAlgoNlRelativeSize{-1}
	\begin{algorithm}[t]
		\begin{RTL}
			\caption{%
				تعویض فدرال
				\lr{(FedSwap)}
				\cite{chiu2020semisupervised}
			}
			\label{algo_FedSwap}
		\end{RTL}
		
		\begin{latin}
			Initialize all clients model with weight $w_0$\;
			\For{$t = 1, 2, \ldots, T$}{
				\For{each client $k = 1, 2, \ldots, K$
					\textbf{in parallel}}{
					$w_t^k = w_{t-1}^k - \eta \nabla F(w_{t-1}^k)$\;
				}
				\If{$t|h_1 = 0$ \quad and \quad $t|h_1h_2 \neq 0$}{
					\For{each client $k = 1, 2, \ldots, K$}{
						$w_t^k \gets \texttt{FedSwap}(k, \{w_t^k\}_{k \in K})$\;
					}
				}
				\If{$t|h_1h_2 = 0$}{
					$w_t \gets \texttt{FedAvg}(\{w_t^k\}_{k \in K})$\;
					\For{each client $k = 1, 2, \ldots, K$
						\textbf{in parallel}}{
						$w_t^k \gets w_t$\;
					}
				}
			}
			\SetKwFunction{FedSwap}{FedSwap}
			\SetKwFunction{FedAvg}{FedAvg}
			\SetKwProg{Fn}{Function}{:}{end}
			\Fn{\FedSwap{$k, \{w_t^k\}_{k \in K}$}}{
				$r$ \space represent a random client in \space $K$\;
				$w_t \gets w_t^r$\;
				$w_t^r \gets w_t^k$\;
				\KwRet $w_t$\;
			}
			\Fn{\FedAvg{$\{w_t^k\}_{k \in K}$}}{
				$w_t \gets \sum_{k=1}^K \frac{n_k}{n} w_t^k$\;
				\KwRet $w_t$\;
			}
		\end{latin}
	\end{algorithm}
\end{LTR}


\begin{table}[h]
	\centering
	\caption{نمادهای مختص الگوریتم
		\lr{FedSwap}
	}
	\label{tabel_FedSwapNotations}
	\begin{tabular}{cr}
		\hline
		متغیر & توضیحات \\
		\hline
		$h_1$ & تعداد گام‌ها بین هر
		\lr{FedSwap} \\
		$h_2$ & تعداد
		\lr{FedSwap}
		بین هر
		\lr{FedAvg} \\
		\hline
	\end{tabular}
\end{table}


برای تعیین مقادیر
$h_1$
و
$h_2$%
، ابتدا آزمایش‌های مختلفی انجام شده و بر اساس نتایج به دست آمده، بهینه‌ترین مقادیر انتخاب شده‌اند. در این آزمایش‌ها، چند نکته مهم مشاهده شده است. ابتدا، مقدار
$h_1$
به عملکرد به‌روزرسانی مدل محلی در دستگاه‌های نهایی وابسته است. از آنجایی که وظیفه یادگیری معمولاً یک وظیفه عمومی مثل طبقه‌بندی است، مقدار
$h_1$
بر اساس مقدار گام تعریف‌شده در روش میانگین‌گیری فدرال سنتی تنظیم می‌شود
\cite{chiu2020semisupervised}.

علاوه بر این، مقدار
$h_2$
نقش مهمی در توازن بین سربار ارتباطی و همگرایی مدل ایفا می‌کند. با افزایش مقدار
$h_2$%
، تعداد دفعات تعویض فدرال بین دستگاه‌های نهایی بیشتر خواهد شد. این امر می‌تواند با کاهش تعداد دفعات ادغام فدرال، پهنای باند ارتباطی بیشتری را صرفه‌جویی کند. با این حال، این امر ممکن است باعث افزایش احتمال انحراف وزن‌ها و کاهش دقت همگرایی مدل جهانی شود. به عبارت دیگر، هرچه مقدار
$h_2$
بزرگتر باشد، تعداد دفعاتی که مدل‌ها بین دستگاه‌های نهایی تعویض می‌شوند بیشتر است و این ممکن است به بهبود عملکرد مدل‌ها در مواجهه با داده‌های
\lr{non-IID}
کمک کند، اما ریسک انحراف وزن‌ها نیز بیشتر خواهد شد.

از سوی دیگر، اگر مقدار
$h_2$
کوچکتر باشد، فراوانی تعویض فدرال بین دستگاه‌های نهایی کاهش می‌یابد. این امر منجر به افزایش سربارهای ارتباطی می‌شود زیرا نیاز به ادغام مکرر مدل جهانی خواهد بود. بنابراین، مقدار
$h_2$
باید به گونه‌ای تنظیم شود که توازن مناسبی بین کاهش سربار ارتباطی و حفظ دقت مدل ایجاد کند.

در مجموع، روش
\lr{FedSwap}
با تعویض مدل‌های محلی بین دستگاه‌های نهایی بدون نیاز به هزینه‌های محاسباتی و ارتباطی اضافی، می‌تواند به بهبود عملکرد مدل‌ها در مواجهه با داده‌های
\lr{non-IID}
کمک کند
\cite{chiu2020semisupervised}.



\section{نحوه جابجایی مدل‌ها}
\subsection{جابجایی تصادفی}
در الگوریتم
\lr{FedSwap}%
، مدل‌ها بین دستگاه‌های نهایی جابجا می‌شوند. انتخاب دستگاه‌های نهایی جهت جابجایی به صورت کاملاً تصادفی انجام می‌گیرد، به این معنا که هنگامی که دو دستگاه می‌خواهند مدل‌های خود را جابجا کنند، فرآیند انتخاب این دو دستگاه به صورت کاملاً تصادفی انجام می‌شود. این تصادفی بودن انتخاب دستگاه‌ها باعث می‌شود که هیچ الگوی ثابتی در جابجایی مدل‌ها وجود نداشته باشد و هر بار ترکیب جدیدی از دستگاه‌ها در فرآیند تبادل مدل شرکت کنند.

یکی از ویژگی‌های مهم الگوریتم
\lr{FedSwap}
این است که تمام دستگاه‌های نهایی به صورت مساوی و عادلانه در این فرآیند جابجایی شرکت می‌کنند. به عبارت دیگر، همه دستگاه‌ها بدون استثنا در فرآیند جابجایی قرار می‌گیرند، اما انتخاب دستگاه‌ها برای جابجایی به صورت تصادفی صورت می‌پذیرد. این روش باعث می‌شود که تمامی دستگاه‌ها فرصت مساوی برای تبادل مدل‌ها و بهبود دقت و عملکرد خود داشته باشند.

با استفاده از این رویکرد، الگوریتم
\lr{FedSwap}
قادر است به بهبود عملکرد مدل‌های محلی کمک کند، زیرا تبادل تصادفی مدل‌ها بین دستگاه‌ها باعث می‌شود که هر دستگاه به داده‌ها و اطلاعات بیشتری دسترسی پیدا کند. این امر به کاهش انحراف وزن‌ها و بهبود همگرایی مدل جهانی کمک می‌کند.


\subsection{جابجایی بر اساس میزان مشابهت مدل‌ها}
در این پژوهش، هدف اصلی انتخاب دستگاه‌های نهایی بر اساس میزان شباهت مدل‌های شبکه عصبی و جابجایی آن‌ها با یکدیگر است. برای این منظور، باید به طور کامل با ساختار شبکه عصبی آشنا بوده و مدل‌های مختلف را با یکدیگر مقایسه کرد. این مقایسه به ما کمک می‌کند تا میزان شباهت و تفاوت بین مدل‌های شبکه عصبی دستگاه‌های نهایی را ارزیابی کنیم.

بعد از بررسی و تعیین میزان شباهت مدل‌ها، باید تصمیم گرفت که کدام یک از آن‌ها را با یکدیگر جابجا کنیم. بهترین انتخاب برای جابجایی، مدلی است که کمترین شباهت را با مدل شبکه عصبی دستگاه فعلی دارد. دلیل این انتخاب این است که اگر مدل دستگاه فعلی با مدل دستگاه مقصد شباهت زیادی داشته باشد، جابجایی آن‌ها مؤثر نخواهد بود. این شباهت بالا به این معناست که این دو دستگاه نهایی داده‌های مشابهی داشته و در طول زمان آموزش‌های مشابهی دیده‌اند، بنابراین جابجایی مدل‌ها تأثیر قابل‌توجهی بر بهبود یادگیری نخواهد داشت.

بنابراین، انتخاب مدل‌هایی با کمترین شباهت بین دستگاه‌های نهایی می‌تواند به بهبود فرآیند یادگیری کمک کند. این فرض بر این اساس است که دستگاه‌هایی با مدل‌های متفاوت احتمالاً داده‌هایی با ساختارهای متفاوت دارند. جابجایی مدل‌ها بین این دستگاه‌ها، مدل‌ها را با داده‌های جدیدی روبه‌رو می‌کند که می‌تواند به یادگیری بهتر و متنوع‌تر کمک کند. در نتیجه، مدل سراسری سریع‌تر به سمت مسیر بهینه همگرا می‌شود و دقت و کارایی آن افزایش می‌یابد.

این روش نه تنها تنوع داده‌ها را در فرآیند یادگیری افزایش می‌دهد، بلکه به کاهش انحراف وزن‌ها نیز کمک می‌کند. با داشتن دید گسترده‌تری از داده‌ها و تجربیات مختلف، مدل‌ها می‌توانند بهتر و جامع‌تر آموزش ببینند. این امر در نهایت منجر به بهبود عملکرد کلی مدل در شرایط واقعی می‌شود و کمک می‌کند که مدل‌های یادگیری فدرال بتوانند با چالش‌های داده‌های
\lr{non-IID}
به نحو بهتری مقابله کنند.

\subsection{بار محاسباتی جهت بررسی مشابهت}

در این روش، ابتدا تمام مدل‌های شبکه عصبی از دستگاه‌های نهایی به سمت سرور ارسال می‌شوند. سپس سرور بر اساس معیارهای مشخصی، شباهت بین این مدل‌ها را بررسی و اقدام به جابجایی آن‌ها می‌کند. در این فرآیند، تمامی محاسبات پردازشی در سرور انجام می‌شود.

افزایش بار کاری در سرور از نظر محاسباتی مشکلی ایجاد نمی‌کند زیرا سرور به‌طور خاص برای انجام چنین عملیات‌هایی طراحی شده است. در یادگیری فدرال، فرض بر این است که دستگاه‌های نهایی دارای سخت‌افزار ضعیف و منابع محدود هستند، بنابراین بار محاسباتی سنگین بر عهده آن‌ها گذاشته نمی‌شود. در این روش نیز تمامی پردازش‌های سنگین بر روی سرور انجام می‌شود که از قدرت پردازشی بالایی برخوردار است و می‌تواند به راحتی این عملیات را مدیریت کند.

سرورهای مرکزی معمولاً دارای منابع پردازشی قوی، حافظه زیاد و قابلیت‌های پیشرفته برای انجام محاسبات پیچیده هستند. این ویژگی‌ها به سرور اجازه می‌دهد تا بدون هیچ مشکلی عملیات‌هایی مانند تعویض مدل‌ها و تجمیع نتایج را انجام دهد. این امر به خصوص در مورد یادگیری فدرال بسیار حیاتی است زیرا بار محاسباتی سنگین نباید بر دستگاه‌های نهایی با سخت‌افزار ضعیف تحمیل شود.

با انجام عملیات بر روی سرور، دستگاه‌های نهایی تنها به تبادل داده‌های لازم و اجرای به‌روزرسانی‌های محلی سبک می‌پردازند. این رویکرد باعث خواهد شد که فرآیند یادگیری بهینه‌تری ایجاد شود و مدل‌ها به طور موثر و کارآمدتری آموزش ببینند. در نتیجه، مشکلات محاسباتی به حداقل می‌رسد و عملکرد کلی سیستم بهبود خواهد یافت.

این روش نه تنها به حفظ منابع محدود دستگاه‌های نهایی کمک می‌کند، بلکه بهره‌وری بالاتری نیز از قدرت پردازشی سرور به دست می‌آید. به این ترتیب، می‌توان اطمینان داشت که عملیات‌های پیچیده و محاسبات سنگین به درستی و با سرعت مناسب انجام می‌شوند، بدون اینکه فشار اضافی بر دستگاه‌های نهایی وارد شود. به این ترتیب، این روش می‌تواند به طور موثری در محیط‌های مختلف با دستگاه‌های متنوع و منابع محدود پیاده‌سازی شود و نتایج قابل اعتمادی ارائه دهد.


\section{معیارهای مشابهت}
\subsection{تعریف مسئله}
فرض کنید \( X \) ماتریسی با ابعاد \( n \times p_1 \) باشد که \( X \in \mathbb{R}^{n \times p_1} \) شامل \( n \) نمونه و \( p_1 \) ویژگی است. همچنین، \( Y \) ماتریسی با ابعاد \( n \times p_2 \) باشد که \( Y \in \mathbb{R}^{n \times p_2} \) شامل \( n \) نمونه و \( p_2 \) ویژگی است. فرض می‌شود که \( p_1 \) کمتر یا مساوی \( p_2 \) است.

هدف طراحی و تحلیل یک شاخص شباهت عددی \( s(X, Y) \) است که بتواند بازنمایی‌های%
\LTRfootnote{Representations}
موجود در ماتریس‌های \( X \) و \( Y \) را هم درون یک شبکه عصبی و هم بین شبکه‌های عصبی مختلف مقایسه کند. چنین شاخصی به ما کمک می‌کند تا تاثیر عوامل مختلف در یادگیری عمیق را بهتر درک کرده و این تاثیرات را به تصویر بکشیم.

به عنوان مثال، در بررسی شبکه‌های عصبی، ماتریس \( X \) می‌تواند نمایانگر فعال‌سازهای%
\LTRfootnote{Activations}
نورون‌ها در یک لایه خاص برای \( n \) نمونه ورودی باشد و ماتریس \( Y \) می‌تواند نمایانگر فعال‌سازهای نورون‌ها در لایه‌ای دیگر یا حتی در یک شبکه عصبی دیگر برای همان \( n \) نمونه باشد. مقایسه این دو ماتریس اطلاعات مهمی درباره نحوه یادگیری و بازنمایی داده‌ها توسط شبکه عصبی ارائه می‌دهد.

شاخص \( s(X, Y) \) باید توانایی اندازه‌گیری شباهت‌ها و تفاوت‌های بین بازنمایی‌های مختلف را داشته باشد. این شاخص می‌تواند به پژوهشگران کمک کند تا نحوه تغییر بازنمایی‌ها در اثر عوامل مختلف مانند تغییرات در داده‌های ورودی، تغییرات در معماری شبکه یا تغییرات در پارامترهای آموزش را بهتر درک کنند.

طراحی و تحلیل این شاخص شباهت می‌تواند به بهبود فهم ما از نحوه عملکرد شبکه‌های عصبی کمک کند و ابزار مفیدی برای بهبود روش‌های آموزش و بهینه‌سازی شبکه‌های عصبی فراهم کند.


\section{معیار مشابهت در چه مواردی می‌بایست پایدار باشد}
این بخش ویژگی‌های لازم برای یک معیار مقایسه بازنمایی‌های شبکه عصبی را بررسی می‌کند. این بررسی شامل تحلیل پایداری شاخص‌های شباهت و تأثیرات آن‌ها در اندازه‌گیری شباهت بازنمایی‌های شبکه عصبی است. اهمیت این موضوع در این است که معیار شباهت مورد استفاده باید نسبت به تبدیل متعامد%
\LTRfootnote{Orthogonal Transformation}
و مقیاس‌بندی همسان‌گرد%
\LTRfootnote{Isotropic Scaling}
پایدار باشد. این ویژگی‌ها به معیار شباهت امکان می‌دهند تا بازنمایی‌های شبکه عصبی را به درستی مقایسه کرده و تأثیرات مختلف در فرآیند آموزش شبکه عصبی را بهتر درک کند.


\subsection{تبدیل متعامد}

تست

\subsection{مقیاس‌بندی همسان‌گرد}
تست





\section{معیارها}
تست

\subsection{sum\_diff}
تست

\subsection{CCA}
تست

\subsection{CKA}
تست

\subsection{dCKA}
تست


\subsection{نحوه پیداکردن مشابهت، لایه به لایه}
تست

\section{نحوه جستجوی کاربر نهایی جهت جابجایی مدل}
تست

\subsection{روش حریصانه}
تست

\subsection{روش بهینه}
تست





