% Chapter 3
\chapter{بررسی پیشینه روش‌های حل مشکل ناهمگنی آماری}

\section{مقدمه}
%همان‌طور که در فصل گذشته اشاره شد، یکی از مهم‌ترین مشکلات در حوزه یادگیری فدرال، مسئله داده‌های غیرمستقل و غیر یکنواخت (non-IID) است. این مشکل منجر به چالش‌هایی در زمینه ناهمگونی آماری می‌شود که حل آن‌ها برای بهبود عملکرد سیستم‌های یادگیری فدرال بسیار حیاتی است. محققان بسیاری در تلاش بوده‌اند تا با ارائه راه‌حل‌های مختلف، این مشکل را برطرف کنند و به نتایج بهتری دست یابند.

%موضوع اصلی این پایان‌نامه نیز به بررسی دقیق همین مشکل و ارائه راه‌حل‌های ممکن برای آن اختصاص دارد. در ادامه، به صورت خلاصه به بررسی راه‌حل‌هایی که تاکنون برای حل این مشکل مطرح شده‌اند، خواهیم پرداخت.


همان‌طور که در فصل گذشته اشاره شد، یکی از مهم‌ترین مشکلات در حوزه یادگیری فدرال، مسئله داده‌های غیرمستقل و غیریکنواخت
\lr{(non-IID)}
است که منجر به بروز چالش‌ها و ناهمگنی‌های آماری می‌شود. این مشکل باعث می‌شود که مدل‌های یادگیری نتوانند به خوبی از داده‌های توزیع‌شده استفاده کنند و کارایی مطلوبی داشته باشند. به دلیل اهمیت بالای این موضوع، محققان بسیاری تلاش‌های گسترده‌ای برای حل این مشکل انجام داده‌اند.

مبحث اصلی این پایان‌نامه نیز به طور دقیق به همین مسئله اشاره دارد و به دنبال یافتن راه‌حلی مؤثر برای مقابله با داده‌های
\lr{non-IID}
است. در ادامه، به صورت خلاصه به بررسی راه‌حل‌هایی که تاکنون برای حل این مشکل مطرح شده‌اند، خواهیم پرداخت تا تصویر جامعی از تلاش‌های انجام شده در این زمینه ارائه دهیم. همچنین باید توجه داشت که هر یک از این راه‌حل‌ها نقاط قوت و ضعف خاص خود را دارند و بسته به شرایط و نوع داده‌ها، می‌توانند نتایج متفاوتی را به همراه داشته باشند. بررسی دقیق این راه‌حل‌ها و ارزیابی کارایی آن‌ها می‌تواند به بهبود سیستم‌های یادگیری فدرال و غلبه بر مشکلات مرتبط با داده‌های غیرمستقل و غیر یکنواخت کمک شایانی کند.


\section{نگرش برپایه داده}
تست

\subsection{تست}
تست

\subsection{تست}
تست



\section{نگرش برپایه مدل}
تست


\section{نگرش برپایه چهارچوب}
تست


\section{نگرش برپایه الگوریتم}
تست
