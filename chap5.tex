% Chapter 5
\chapter{پیاده‌سازی و بررسی نتایج}

\section{مقدمه}
تست


\section{انواع مجموعه داده}
تست

\subsection{
مجموعه داده
	\lr{MNIST}%
	\LTRfootnote{Modified National Institute of Standards and Technology}
}
مجموعه داده
\lr{MNIST}
یکی از مشهورترین و پر استفاده‌ترین مجموعه داده‌ها در زمینه یادگیری ماشین است. این مجموعه شامل تصاویر دست‌نویس از اعداد 0 تا 9 می‌باشد و به طور گسترده‌ای برای آموزش و ارزیابی مدل‌های مختلف یادگیری ماشین به کار می‌رود.
مجموعه داده
\lr{MNIST}
در دهه 1990 توسط یان لکون%
\LTRfootnote{Yann LeCun}،
کورینا کورتس
\LTRfootnote{Corinna Cortes}
و کریستوفر برجس%
\LTRfootnote{Christopher Burges}
 ایجاد شد. هدف اصلی این مجموعه داده، فراهم کردن یک مجموعه استاندارد برای ارزیابی الگوریتم‌های یادگیری ماشین و بینایی کامپیوتر بود.

مجموعه داده
\lr{MNIST}
شامل 70٬000 تصویر از ارقام دست‌نویس است که به دو بخش شامل مجموعه آموزش با 60٬000 تصویر و مجموعه تست با 10٬000 تصویر تقسیم می‌شود. هر تصویر دارای ابعاد
(28$\times$28)
پیکسل است که به صورت خاکستری%
\LTRfootnote{Grayscale}
ذخیره شده‌اند و هر پیکسل دارای مقداری بین 0 (سیاه) تا 255 (سفید) است. همچنین تمامی تصاویر با یک برچسب عددی بین 0 تا 9 همراه هستند که نمایانگر رقم موجود در تصویر می‌باشد
\cite{lecun1998gradient}.

داده‌ها معمولاً در قالب دو فایل باینری شامل یکی برای تصاویر و دیگری برای برچسب‌ها ذخیره می‌شوند. هر تصویر به صورت یک بردار از اعداد بین 0 تا 255 با طول 784
(28$\times$28)
ذخیره می‌شود. به دلیل یکنواختی تصاویر و اندازه کوچک آن‌ها، نیاز به پیش‌پردازش پیچیده‌ای ندارند. یکی از مراحل پیش‌پردازش شامل نرمال‌سازی یا همان تبدیل مقادیر پیکسل‌ها به مقادیر بین 0 و 1 می‌باشد.

مجموعه داده
\lr{MNIST}
به عنوان یک نقطه شروع استاندارد برای آموزش و ارزیابی مدل‌های مختلف یادگیری عمیق و شبکه‌های عصبی استفاده می‌شود. محققان اغلب از
\lr{MNIST}
برای مقایسه کارایی الگوریتم‌های جدید با الگوریتم‌های موجود استفاده می‌کنند. این مجموعه شامل نمونه‌های متنوعی از ارقام دست‌نویس از افراد مختلف است که موجب می‌شود به عنوان یک معیار استاندارد برای مقایسه مدل‌ها و الگوریتم‌ها مورد استفاده قرار گیرد.

مجموعه داده
\lr{MNIST}
دارای مزایای زیادی از جمله سادگی، در دسترس بودن، استاندارد بودن و پراکندگی داده‌ها می‌باشد. با این حال، این مجموعه داده دارای معایبی نیز هست. به عنوان مثال، برای مسائل پیچیده‌تر و واقعی‌تر ممکن است
\lr{MNIST}
خیلی ساده باشد و نتواند چالش‌های واقعی را نشان دهد. همچنین، این مجموعه داده شامل تنها اعداد 0 تا 9 است و برای سایر کاربردهای دسته‌بندی تصویر ممکن است کافی نباشد.

کاربردهای عملی این مجموعه داده شامل آموزش شبکه‌های عصبی متفاوت برای بهبود دقت دسته‌بندی، تست و ارزیابی مدل‌های مختلف یادگیری عمیق و الگوریتم‌های بهینه‌سازی است. بسیاری از مدل‌ها و الگوریتم‌های پیشرفته امروزی با استفاده از مجموعه داده
\lr{MNIST}
توسعه و ارزیابی شده‌اند.

به طور کلی، مجموعه داده
\lr{MNIST}
با توجه به دلایل ذکر شده، یکی از مهم‌ترین و پراستفاده‌ترین مجموعه داده‌ها در زمینه یادگیری ماشین و بینایی کامپیوتر است. این مجموعه به محققان و دانشجویان کمک می‌کند تا مفاهیم پایه‌ای یادگیری ماشین را به خوبی درک کرده و الگوریتم‌های جدید را ارزیابی کنند.


\subsection{
	مجموعه داده
	\lr{CIFAR-10}%
	\LTRfootnote{Canadian Institute For Advanced Research}
}
مجموعه داده
\lr{CIFAR-10}
یکی از معروف‌ترین و پرکاربردترین مجموعه داده‌های مورد استفاده در حوزه یادگیری ماشین و بینایی کامپیوتر است. این مجموعه داده توسط گروهی به سرپرستی الکس کریژفسکی%
\LTRfootnote{Alex Krizhevsky}
و جفری هینتون%
\LTRfootnote{Geoffrey Hinton}
در دانشگاه تورنتو گردآوری شده و برای ارزیابی و آزمایش مدل‌های یادگیری عمیق به کار می‌رود.


مجموعه داده
\lr{CIFAR-10}
شامل 60٬000 تصویر رنگی با اندازه
32$\times$32
پیکسل است که به 10 کلاس مختلف تقسیم شده‌اند. هر کلاس شامل 6٬000 تصویر است که به صورت مساوی بین مجموعه‌های آموزشی و آزمایشی توزیع شده‌اند. این کلاس‌ها شامل مواردی مانند هواپیما، اتومبیل، پرنده، گربه، گوزن، سگ، قورباغه، اسب، کِشتی و کامیون هستند. هر یک از این کلاس‌ها دارای تصاویری است که تنوع بالایی از زوایا، پس‌زمینه‌ها و شرایط نوری مختلف را شامل می‌شود.

یکی از ویژگی‌های مهم مجموعه داده
\lr{CIFAR-10}
تنوع بالای تصاویر در هر کلاس است. این تنوع باعث می‌شود که مدل‌های یادگیری عمیق نیاز به توانایی تعمیم‌دهی بالا برای تشخیص صحیح کلاس‌ها داشته باشند. این مجموعه داده برای آموزش و ارزیابی مدل‌های مختلفی مورد استفاده قرار می‌گیرد و بسیاری از پژوهش‌ها و مقالات علمی از آن به عنوان مبنای مقایسه عملکرد مدل‌ها استفاده کرده‌اند.

مجموعه داده
\lr{CIFAR-10}
به دو بخش آموزشی و آزمایشی تقسیم شده است. بخش آموزشی شامل 50٬000 تصویر و بخش آزمایشی شامل 10٬000 تصویر است. این تقسیم‌بندی، استانداردی برای ارزیابی مدل‌ها فراهم می‌کند، به طوری که مدل‌ها می‌توانند بر روی مجموعه آموزشی، آموزش دیده و سپس بر روی مجموعه آزمایشی ارزیابی شوند. این روش به محققان امکان می‌دهد تا عملکرد مدل‌ها را به صورت عینی و قابل تکرار مقایسه کنند.

به دلیل اندازه کوچک تصاویر (%
32$\times$32
پیکسل)، پردازش و آموزش مدل‌ها بر روی
\lr{CIFAR-10}
نسبتاً سریع و کم هزینه است. این ویژگی باعث شده تا مجموعه داده 
\lr{CIFAR-10}
برای آزمایش مدل‌ها بسیار مناسب باشد. بسیاری از ابزارها و چارچوب‌های%
\LTRfootnote{Frameworks}
یادگیری ماشین مانند
\lr{PyTorch}
و
\lr{TensorFlow}
شامل توابع و ابزارهای آماده برای بارگذاری و استفاده از این مجموعه داده هستند که این امر نیز به سهولت استفاده از آن کمک می‌کند.

در نهایت، مجموعه داده
\lr{CIFAR-10}
با ارائه تصاویری متنوع و چالش‌برانگیز در کلاس‌های مختلف، ابزاری قدرتمند برای آموزش و ارزیابی مدل‌های یادگیری عمیق فراهم می‌کند. این مجموعه داده نه تنها در پژوهش‌های دانشگاهی بلکه در صنعت نیز به عنوان معیاری برای ارزیابی پیشرفت‌ها در حوزه بینایی کامپیوتر استفاده می‌شود.


\subsection{
	مجموعه داده
	\lr{CINIC-10}%
	\LTRfootnote{CIFAR-10 and ImageNet Combined}
}
مجموعه داده
\lr{CINIC-10}
یک مجموعه داده تصویری گسترده و متنوع است که برای ارزیابی عملکرد مدل‌های یادگیری ماشین به ویژه در زمینه‌های مرتبط با طبقه‌بندی تصاویر مورد استفاده قرار می‌گیرد. این مجموعه داده ترکیبی از تصاویر موجود در مجموعه‌ داده‌های معروف
\lr{CIFAR-10}
و
\lr{ImageNet}
است. این ترکیب به منظور ایجاد مجموعه‌ای گسترده‌تر و متنوع‌تر از تصاویر انجام شده است که می‌تواند به ارزیابی دقیق‌تر و واقع‌گرایانه‌تر مدل‌ها کمک کند.

مجموعه داده
\lr{CINIC-10}
شامل 270٬000 تصویر است که در 10 کلاس مختلف دسته‌بندی شده‌اند. هر کلاس شامل 27٬000 تصویر است که به دو بخش آموزشی و آزمایشی تقسیم شده‌اند. بخش آموزشی شامل 180٬000 تصویر و بخش آزمایشی شامل 90٬000 تصویر است. این تقسیم‌بندی منظم به محققان و مهندسان یادگیری ماشین این امکان را می‌دهد که به راحتی مدل‌های خود را آموزش داده، اعتبارسنجی و آزمایش کنند.

%تصاویر موجود در CINIC-10 دارای ابعاد 32x32 پیکسل هستند، که مشابه ابعاد تصاویر موجود در CIFAR-10 است. این ویژگی باعث می‌شود که مدل‌های از پیش آموزش دیده بر روی CIFAR-10 بتوانند به راحتی بر روی این مجموعه داده نیز مورد استفاده قرار گیرند و ارزیابی شوند. با این حال، تنوع بیشتر تصاویر در CINIC-10 نسبت به CIFAR-10 به دلیل ترکیب تصاویر از ImageNet، چالشی جدی‌تر برای مدل‌های یادگیری ماشین فراهم می‌کند.
%
%یکی از اهداف اصلی ایجاد CINIC-10، افزایش تنوع و پیچیدگی تصاویر مورد استفاده برای آموزش و ارزیابی مدل‌ها بود. این مجموعه داده شامل تصاویری از دنیای واقعی است که در شرایط نوری مختلف و با پس‌زمینه‌های متنوع گرفته شده‌اند. این ویژگی به مدل‌ها کمک می‌کند تا به جای اینکه تنها بر روی مجموعه‌ای محدود از تصاویر آموزش ببینند، توانایی تعمیم‌دهی خود را به تصاویر جدید و غیرمنتظره نیز افزایش دهند.
%
%در نهایت، CINIC-10 با هدف ارتقای استانداردهای ارزیابی مدل‌های یادگیری عمیق و بهبود عملکرد آنها در مواجهه با داده‌های واقعی و متنوع ایجاد شده است. این مجموعه داده به محققان این امکان را می‌دهد که مدل‌های خود را در شرایط نزدیک به دنیای واقعی آزمایش کرده و نقاط ضعف و قوت آنها را بهتر شناسایی کنند. به همین دلیل، CINIC-10 به عنوان یک ابزار ارزشمند در جامعه یادگیری ماشین شناخته می‌شود و به طور گسترده‌ای مورد استفاده قرار می‌گیرد.


\subsection{FEMNIST}
تست


\section{پیاده‌سازی مدل‌های شبکه‌عصبی}
تست

\subsection{MLP}
تست

\subsection{CNN}
تست



\section{بررسی نتایج}
تست



