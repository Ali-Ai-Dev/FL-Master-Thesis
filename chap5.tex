% Chapter 5
\chapter{پیاده‌سازی و بررسی نتایج}

\section{مقدمه}
تست


\section{انواع مجموعه داده}
تست

\subsection{
مجموعه داده
	\lr{MNIST}%
	\LTRfootnote{Modified National Institute of Standards and Technology}
}
مجموعه داده
\lr{MNIST}
یکی از مشهورترین و پر استفاده‌ترین مجموعه داده‌ها در زمینه یادگیری ماشین است. این مجموعه شامل تصاویر دست‌نویس از اعداد 0 تا 9 می‌باشد و به طور گسترده‌ای برای آموزش و ارزیابی مدل‌های مختلف یادگیری ماشین به کار می‌رود.
مجموعه داده
\lr{MNIST}
در دهه 1990 ایجاد شد. هدف اصلی این مجموعه داده، فراهم کردن یک مجموعه استاندارد برای ارزیابی الگوریتم‌های یادگیری ماشین و بینایی کامپیوتر بود.

مجموعه داده
\lr{MNIST}
شامل 70,000 تصویر از ارقام دست‌نویس است که به دو بخش شامل مجموعه آموزش با 60,000 تصویر و مجموعه تست با 10,000 تصویر تقسیم می‌شود. هر تصویر دارای ابعاد
(28$\times$28)
پیکسل است که به صورت خاکستری%
\LTRfootnote{Grayscale}
ذخیره شده‌اند و هر پیکسل دارای مقداری بین 0 (سیاه) تا 255 (سفید) است. همچنین تمامی تصاویر با یک برچسب عددی بین 0 تا 9 همراه هستند که نمایانگر رقم موجود در تصویر می‌باشد.

داده‌ها معمولاً در قالب دو فایل باینری شامل یکی برای تصاویر و دیگری برای برچسب‌ها ذخیره می‌شوند. هر تصویر به صورت یک بردار از اعداد بین 0 تا 255 با طول 784
(28$\times$28)
ذخیره می‌شود. به دلیل یکنواختی تصاویر و اندازه کوچک آن‌ها، نیاز به پیش‌پردازش پیچیده‌ای ندارند. یکی از مراحل پیش‌پردازش شامل نرمال‌سازی یا همان تبدیل مقادیر پیکسل‌ها به مقادیر بین 0 و 1 می‌باشد.
%
%مجموعه داده MNIST به عنوان یک نقطه شروع استاندارد برای آموزش و ارزیابی مدل‌های مختلف یادگیری عمیق و شبکه‌های عصبی استفاده می‌شود. محققان اغلب از MNIST برای مقایسه کارایی الگوریتم‌های جدید با الگوریتم‌های موجود استفاده می‌کنند. همچنین بسیاری از دوره‌های آموزشی و منابع آنلاین از MNIST برای آموزش مفاهیم پایه‌ای یادگیری ماشین و شبکه‌های عصبی استفاده می‌کنند. این مجموعه شامل نمونه‌های متنوعی از ارقام دست‌نویس از افراد مختلف است که موجب می‌شود به عنوان یک معیار استاندارد برای مقایسه مدل‌ها و الگوریتم‌ها مورد استفاده قرار گیرد.
%
%مجموعه داده MNIST دارای مزایای زیادی است از جمله سادگی و در دسترس بودن، استاندارد بودن، و پراکندگی داده‌ها. با این حال، این مجموعه داده‌ها دارای معایبی نیز می‌باشد. به عنوان مثال، برای مسائل پیچیده‌تر و واقعی‌تر ممکن است MNIST خیلی ساده باشد و نتواند چالش‌های واقعی را نشان دهد. همچنین، این مجموعه داده‌ها شامل تنها اعداد 0 تا 9 است و برای سایر کاربردهای دسته‌بندی تصویر ممکن است کافی نباشد.
%
%کاربردهای عملی این مجموعه داده شامل آموزش یک شبکه عصبی ساده (MLP) برای دسته‌بندی ارقام، استفاده از شبکه‌های عصبی پیچیده‌تر مانند CNN (شبکه‌های عصبی کانولوشنی) برای بهبود دقت دسته‌بندی، و تست و ارزیابی مدل‌های مختلف یادگیری عمیق و الگوریتم‌های بهینه‌سازی است. بسیاری از مدل‌ها و الگوریتم‌های پیشرفته امروزی با استفاده از مجموعه داده MNIST توسعه و ارزیابی شده‌اند. دقت مدل‌ها بر روی MNIST به عنوان یک معیار مهم در مقالات علمی و تحقیقاتی ذکر می‌شود.
%
%به طور کلی، مجموعه داده MNIST به دلیل سادگی، دسترسی آسان و استاندارد بودن، یکی از مهم‌ترین و پراستفاده‌ترین مجموعه داده‌ها در زمینه یادگیری ماشین و بینایی کامپیوتر است. این مجموعه به محققان و دانشجویان کمک می‌کند تا مفاهیم پایه‌ای یادگیری ماشین را به خوبی درک کرده و الگوریتم‌های جدید را ارزیابی کنند.



\subsection{CIFAR-10}
تست

\subsection{CINIC-10}
تست

\subsection{FEMNIST}
تست


\section{پیاده‌سازی مدل‌های شبکه‌عصبی}
تست

\subsection{MLP}
تست

\subsection{CNN}
تست



\section{بررسی نتایج}
تست



