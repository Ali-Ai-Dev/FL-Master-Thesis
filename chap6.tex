% Chapter 6
\chapter{نتیجه‌گیری و پیشنهاد‌ها}

\section{نتیجه‌گیری}

در دنیای امروزی با رشد سریع فناوری و افزایش تعداد دستگاه‌های متصل به اینترنت، نیاز به برقراری ارتباطات مؤثر و حفاظت از حریم شخصی کاربران بیش از پیش اهمیت یافته است. این مسئله منجر به توسعه روش‌های توزیع‌شده‌ای مانند یادگیری فدرال شده است. در یادگیری فدرال، داده‌ها به‌جای ارسال به یک سرور مرکزی برای پردازش، در محل خود دستگاه‌ها نگهداری می‌شوند و مدل‌ها به‌صورت محلی آموزش داده می‌شوند. سپس این مدل‌ها با یکدیگر ترکیب می‌شوند تا یک مدل سراسری به‌دست آید. این روش علاوه بر کاهش نیاز به انتقال داده‌ها، حریم شخصی کاربران را نیز بهتر حفظ می‌کند.

با این حال، یادگیری فدرال با چالش‌های متعددی مواجه است. یکی از این چالش‌ها، ناهمگنی آماری داده‌ها است. به این معناست که داده‌های موجود در دستگاه‌های مختلف می‌توانند بسیار متفاوت و گوناگون باشند. این ناهمگنی باعث می‌شود که مدل‌های محلی نتوانند به‌طور کامل ویژگی‌های تمامی داده‌ها را یاد بگیرند و در نتیجه، مدل سراسری نیز به‌خوبی همگرا نشود. در این شرایط، رسیدن به یک مدل سراسری که عملکرد قابل قبولی داشته باشد، ممکن است با مشکل مواجه شود.

برای مقابله با این چالش، یکی از راهکارهای پیشنهادی، جابه‌جایی وزن‌های شبکه عصبی بین کاربران نهایی در طول فرآیند یادگیری است. این روش باعث می‌شود که مدل‌های محلی با داده‌های متنوع‌تری روبه‌رو شوند و در نتیجه، مدل سراسری بتواند به یک همگرایی بهتر دست یابد. به عبارت دیگر، با جابه‌جایی وزن‌ها، مدل‌ها به داده‌های بیشتری دسترسی پیدا می‌کنند که این امر به بهبود کیفیت یادگیری و همگرایی مدل کمک می‌کند.

در روش‌های متداول، جابه‌جایی وزن‌ها به‌طور تصادفی انجام می‌شود، اما در این پژوهش پیشنهاد شده است که به‌جای استفاده از روش تصادفی، این جابه‌جایی به‌صورت هوشمندانه و بر اساس معیارهای شباهت انجام شود. به این صورت که مدل‌هایی که کمترین میزان شباهت را با یکدیگر دارند، جابه‌جا شوند. این رویکرد باعث می‌شود که مدل‌ها با داده‌هایی که به‌طور کامل با آن‌ها آشنا نیستند، روبه‌رو شوند و این امر می‌تواند به بهبود همگرایی مدل سراسری منجر شود. به‌عبارت دیگر، این نوع جابه‌جایی هوشمندانه می‌تواند مدل سراسری را سریع‌تر و بهتر به همگرایی نهایی هدایت کند.

یکی دیگر از جنبه‌های این پژوهش، بررسی تأثیر جابه‌جایی وزن‌ها بر حفظ حریم شخصی کاربران است. روش‌های مرسوم جابه‌جایی، این فرآیند را به‌طور مستقیم بین کاربران نهایی انجام می‌دهند. این روش اگرچه می‌تواند به کاهش سربار شبکه کمک کند، اما ممکن است به تشدید مشکلات حریم شخصی منجر شود. در این پژوهش، پیشنهاد شده است که سرور مرکزی به عنوان واسطه در فرآیند جابه‌جایی عمل کند. با این کار، حریم شخصی کاربران بهتر حفظ می‌شود و پیاده‌سازی روش‌هایی مانند حفظ حریم خصوصی تفاضلی نیز ساده‌تر خواهد شد.

باید توجه داشت که هرچند این روش بهبود قابل‌توجهی در حفظ حریم شخصی ایجاد می‌کند، اما نسبت به روش‌های جابه‌جایی وزن‌ها بدون دخالت سرور، اندکی به سربار شبکه می‌افزاید. با این حال، این سربار در مقایسه با روش‌های سنتی مانند میانگین‌گیری فدرال همچنان بسیار کمتر خواهد بود. در واقع، مزایایی که این روش در بهبود همگرایی و حفاظت از حریم شخصی به ارمغان می‌آورد، به‌خوبی افزایش جزئی سربار شبکه را نسبت به روش‌های جابه‌جایی بدون دخالت سرور جبران می‌کند و آن را به یک گزینه مطلوب و کارآمد تبدیل کرده است.

در نهایت، این پژوهش نشان می‌دهد که جابه‌جایی مدل‌های شبکه عصبی به‌صورت هوشمندانه و بر اساس معیارهای شباهت، می‌تواند فرآیند همگرایی مدل سراسری را تسریع کند. این امر به‌ویژه در شرایطی که تعداد کاربران زیاد باشد، اثرات مثبت خود را بهتر نشان می‌دهد. به بیان دیگر، هرچه تعداد کاربران بیشتر باشد، مزایای استفاده از این روش بیشتر خواهد بود و بهبود قابل توجهی در فرآیند همگرایی مشاهده می‌شود.

بنابراین، در مواردی که تعداد کاربران زیاد است و داده‌ها بسیار متفاوت و پراکنده هستند، استفاده از روش‌ جابه‌جایی وزن‌ها بر پایه شباهت می‌تواند بهبود قابل ملاحظه‌ای در کارایی و سرعت همگرایی مدل سراسری به‌همراه داشته باشد. این روش نه‌تنها از نظر فنی مؤثر است، بلکه از دیدگاه حفاظت از حریم شخصی نیز بسیار مناسب به‌نظر می‌رسد.

در نتیجه، این پژوهش اهمیت و تأثیر جابه‌جایی هوشمندانه وزن‌های شبکه‌های عصبی در یادگیری فدرال را برجسته می‌کند و راهکاری جدید برای بهبود فرآیند همگرایی و حفظ حریم شخصی کاربران ارائه می‌دهد. این رویکرد می‌تواند به‌عنوان یک راه‌حل مؤثر در مواجهه با چالش‌های موجود در یادگیری فدرال مورد استفاده قرار گیرد.



\section{پیشنهادها}
با توجه به رویکردهای بررسی‌شده در زمینه حل چالش‌های یادگیری فدرال، پیشنهادهای زیر برای ادامه این پژوهش مطرح می‌گردند:

\begin{enumerate}
	\item
	بررسی معیارهای مختلف و انتخاب کمینه شباهت
	
	در آغاز کار، یک معیار شباهت در سرور مشخص می‌شود و تمام مقایسه‌های شبکه‌های عصبی بر اساس این معیار انجام می‌گیرد. با این حال، می‌توان چند معیار را برای ارزیابی شبکه‌های عصبی مد نظر قرار داد. به عنوان نمونه، ماتریس بالا ‌مثلثی برای همه معیارها محاسبه می‌شود و در مرحله انتخاب دو مدل برای جابه‌جایی، مقدار کمینه از میان تمام معیارها برگزیده می‌شود.
	
	
	\item
	میانگین وزن‌دار بین لایه‌های مختلف
	
هنگام مقایسه دو مدل شبکه عصبی، این مقایسه به‌صورت لایه به لایه انجام شده و در نهایت میانگین تمامی لایه‌ها محاسبه می‌شود. سوالی که مطرح می‌شود این است که آیا تأثیر تمام لایه‌ها به یک اندازه است که از میانگین‌گیری با وزن یکسان استفاده می‌شود؟ در این‌جا، می‌توان اهمیت هر لایه در شبکه عصبی را تعیین کرد و سپس میانگین وزن‌دار را برای مقایسه به کار برد.
	

	\item
بررسی شباهت بین لایه‌های مختلف دو شبکه عصبی، نه فقط لایه‌های متناظر

در روش موجود، مقایسه دو شبکه عصبی تنها با بررسی لایه‌های متناظر و محاسبه معیار شباهت انجام می‌گردد. با این حال، از آن‌جا که در شبکه‌های عصبی ممکن است بین لایه‌های مختلف نیز ارتباط وجود داشته باشد، امکان مقایسه لایه‌های غیرمتناظر نیز وجود دارد. از طریق ارزیابی این لایه‌ها، می‌توان شاخصی مناسب برای تعیین شباهت کلی شبکه‌ها، استخراج کرد.
\end{enumerate}

