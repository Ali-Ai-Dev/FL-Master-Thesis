%*************************************************
% In this file the abstract is typeset.
% Make changes accordingly.
%*************************************************

\addcontentsline{toc}{section}{چکیده}
\newgeometry{left=2.5cm,right=3cm,top=3cm,bottom=2.5cm,includehead=false,headsep=1cm,footnotesep=.5cm}
\setcounter{page}{1}
\pagenumbering{arabic}						% شماره صفحات با عدد
\thispagestyle{empty}

~\vfill

\subsection*{چکیده}
\begin{small}
\baselineskip=0.7cm

در عصر حاضر، با پیشرفت سریع فناوری، افزایش تعداد دستگاه‌های متصل به اینترنت و نقش فزاینده هوش مصنوعی و یادگیری ماشین، اهمیت برقراری ارتباطات مؤثر و حفاظت از حریم شخصی کاربران بیش از پیش احساس می‌شود. این ضرورت به توسعه روش‌های توزیع‌شده‌ای مانند یادگیری فدرال انجامیده است که از جمله راهکارهای پیشرفته در این حوزه به‌شمار می‌رود. در یادگیری فدرال، داده‌ها به‌جای ارسال به یک سرور مرکزی، در همان دستگاه‌های نهایی باقی می‌مانند و مدل‌ها به‌صورت محلی آموزش داده می‌شوند. سپس این مدل‌ها با هم ترکیب می‌شوند تا یک مدل جامع ایجاد شود. این روش نه تنها نیاز به انتقال داده‌ها را کاهش می‌دهد، بلکه به حفظ بهتر حریم شخصی کاربران نیز کمک می‌کند.
با این حال، یادگیری فدرال با چالش‌های زیادی روبه‌رو است که یکی از آن‌ها ناهمگنی آماری داده‌ها می‌باشد، به این معنی که داده‌های موجود در دستگاه‌های مختلف می‌توانند بسیار متنوع و متفاوت از یکدیگر باشند. این ناهمگنی باعث می‌شود که مدل‌های محلی نتوانند تمامی ویژگی‌های داده‌ها را به‌خوبی یاد بگیرند و در نتیجه، مدل جامع نیز به خوبی همگرا نشود. بنابراین، دستیابی به یک مدل جامع با عملکرد مناسب ممکن است دشوار شود. در این راستا، ارائه روش‌هایی برای مقابله با ناهمگنی آماری از اهمیت بالایی برخوردار هستند. روش‌های پیشنهادی باید علاوه بر تمرکز بر حل این مشکل، از جنبه‌های محاسباتی، ارتباطی و حفظ حریم شخصی نیز پایداری خود را حفظ کنند.
یکی از راهکارهای پیشنهادی برای مقابله با این چالش، جابه‌جایی مدل‌های شبکه عصبی بین کاربران نهایی در طول فرآیند یادگیری است. این کار باعث می‌شود مدل‌های محلی با داده‌های متنوع‌تری مواجه شوند و در نتیجه، مدل جامع به همگرایی بهتری برسد. در روش‌های معمول، جابه‌جایی مدل‌ها به‌صورت تصادفی انجام می‌شود. اما در این پژوهش پیشنهاد شده است که به‌جای روش تصادفی، این جابه‌جایی به‌صورت هوشمند و بر اساس معیارهای شباهت صورت گیرد. به این ترتیب، مدل‌هایی که کمترین شباهت را با هم دارند، جابه‌جا می‌شوند. این رویکرد باعث می‌شود مدل‌ها با داده‌هایی روبه‌رو شوند که کمتر با آن‌ها آشنا هستند و این امر می‌تواند به بهبود همگرایی مدل جامع منجر شود.
از جنبه دیگر، این پژوهش به بررسی تأثیر جابه‌جایی مدل‌ها بر حفظ حریم شخصی کاربران پرداخته است. روش‌های معمول جابه‌جایی مدل‌ها، به‌طور مستقیم بین کاربران نهایی انجام می‌شوند. اگرچه این روش می‌تواند سربار شبکه را کاهش دهد، اما ممکن است به تضعیف حریم شخصی کاربران منجر شود. در این پژوهش پیشنهاد شده است که سرور مرکزی به عنوان واسطه‌ای در فرآیند جابه‌جایی عمل کند. با این روش، حفظ حریم شخصی کاربران بهتر تضمین می‌شود و پیاده‌سازی تکنیک‌های مختلف این حوزه نیز ساده‌تر خواهد شد.
در نهایت، این پژوهش نشان می‌دهد که جابه‌جایی هوشمندانه مدل‌های شبکه عصبی بر اساس معیارهای شباهت، می‌تواند فرآیند همگرایی مدل جامع را تسریع کند. این روش به‌ویژه در شرایطی که تعداد کاربران زیاد است، تاثیر بیشتری دارد و توانسته است نتایج را حدود
{\footnotesize \(\%\)}%
1 بهبود بخشد.



\vspace{5mm}
\noindent\textbf{کلمات کلیدی:}
1- یادگیری فدرال،
2- یادگیری توزیع‌شده،
3- یادگیری عمیق،
4- شباهت در شبکه عصبی،
5- ناهمگنی آماری
\end{small}