% Appendix 1
\chapter{بررسی نمودارهای خطا}

\section{
	مقایسه روش
	\lr{\texttt{\fontspec{Times New Roman} SimFedSwap}}
	با روش‌های پایه
}


\subsection{
	مجموعه داده
	\lr{MNIST}
}

\begin{figure}[h]
	\centering
	\subfigure[
	دید کلی از نتیجه
	\qquad\hspace{3mm}]{
		\label{app_result_mnist_mlp_mid}
		\includegraphics*[width=.47\textwidth]{images/chap5/result/mnist/loss_mid_mlp.png}
	}
	\hspace{1.5mm}
	\subfigure[
	بزرگ‌نمایی شده بخش اصلی					
	\qquad\hspace{5mm}]{
		\label{app_result_mnist_mlp_zoom}
		\includegraphics*[width=.47\textwidth]{images/chap5/result/mnist/loss_zoom_mlp.png}
	}
	\caption{
		مقایسه منحنی‌های خطا بر روی مجموعه داده
		\lr{MNIST}
		با استفاده از مدل
		\lr{MLP}.
	}
	\label{app_result_mnist_mlp}
\end{figure}



\begin{figure}[h]
	\centering
	\subfigure[
	دید کلی از نتیجه
	\qquad\hspace{3mm}]{
		\label{app_result_mnist_cnn_mid}
		\includegraphics*[width=.47\textwidth]{images/chap5/result/mnist/loss_mid_cnn.png}
	}
	\hspace{1.5mm}
	\subfigure[
	بزرگ‌نمایی شده بخش اصلی					
	\qquad\hspace{5mm}]{
		\label{app_result_mnist_cnn_zoom}
		\includegraphics*[width=.47\textwidth]{images/chap5/result/mnist/loss_zoom_cnn.png}
	}
	\caption{
		مقایسه منحنی‌های خطا بر روی مجموعه داده
		\lr{MNIST}
		با استفاده از مدل
		\lr{CNN}.
	}
	\label{app_result_mnist_cnn}
\end{figure}





\FloatBarrier
\subsection{
	مجموعه داده
	\lr{CIFAR-10}
}


\begin{figure}[h]
	\centering
	\subfigure[
	دید کلی از نتیجه
	\qquad\hspace{3mm}]{
		\label{app_result_cifar10_equal_mid}
		\includegraphics*[width=.48\textwidth]{images/chap5/result/cifar10/loss_mid_equal.png}
	}
	\hspace{0.8mm}
	\subfigure[
	بزرگ‌نمایی شده بخش اصلی					
	\qquad\hspace{5mm}]{
		\label{app_result_cifar10_equal_zoom}
		\includegraphics*[width=.48\textwidth]{images/chap5/result/cifar10/loss_zoom_equal.png}
	}
	\caption{
		مقایسه منحنی‌های خطا بر روی مجموعه داده
		\lr{CIFAR-10}
		با توزیع داده یکنواخت.
	}
	\label{app_result_cifar10_equal}
\end{figure}



\begin{figure}[h]
	\centering
	\subfigure[
	دید کلی از نتیجه
	\qquad\hspace{3mm}]{
		\label{app_result_cifar10_normal_base}
		\includegraphics*[width=.48\textwidth]{images/chap5/result/cifar10/loss_base_normal.png}
	}
	\hspace{0.8mm}
	\subfigure[
	بزرگ‌نمایی شده بخش اصلی					
	\qquad\hspace{5mm}]{
		\label{app_result_cifar10_normal_zoom}
		\includegraphics*[width=.48\textwidth]{images/chap5/result/cifar10/loss_zoom_normal.png}
	}
	\caption{
		مقایسه منحنی‌های خطا بر روی مجموعه داده
		\lr{CIFAR-10}
		با توزیع داده نرمال.
	}
	\label{app_result_cifar10_normal}
\end{figure}



\FloatBarrier
\subsection{
	مجموعه داده
	\lr{CINIC-10}
}



\subsection{
	مجموعه داده
	\lr{FEMNIST}
}



\section{
	مقایسه جابه‌جایی حریصانه با جابه‌جایی حداقل شباهت در روش
	\lr{\texttt{\fontspec{Times New Roman} SimFedSwap}}
}
تست


\section{
	تحلیل کاهش تعداد کاربران در هر دور و افزایش تعداد کل دورها در روش
	\lr{\texttt{\fontspec{Times New Roman} SimFedSwap}}
}
تست
